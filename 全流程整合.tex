% Options for packages loaded elsewhere
\PassOptionsToPackage{unicode}{hyperref}
\PassOptionsToPackage{hyphens}{url}
%
\documentclass[
]{article}
\usepackage{amsmath,amssymb}
\usepackage{iftex}
\ifPDFTeX
  \usepackage[T1]{fontenc}
  \usepackage[utf8]{inputenc}
  \usepackage{textcomp} % provide euro and other symbols
\else % if luatex or xetex
  \usepackage{unicode-math} % this also loads fontspec
  \defaultfontfeatures{Scale=MatchLowercase}
  \defaultfontfeatures[\rmfamily]{Ligatures=TeX,Scale=1}
\fi
\usepackage{lmodern}
\ifPDFTeX\else
  % xetex/luatex font selection
\fi
% Use upquote if available, for straight quotes in verbatim environments
\IfFileExists{upquote.sty}{\usepackage{upquote}}{}
\IfFileExists{microtype.sty}{% use microtype if available
  \usepackage[]{microtype}
  \UseMicrotypeSet[protrusion]{basicmath} % disable protrusion for tt fonts
}{}
\makeatletter
\@ifundefined{KOMAClassName}{% if non-KOMA class
  \IfFileExists{parskip.sty}{%
    \usepackage{parskip}
  }{% else
    \setlength{\parindent}{0pt}
    \setlength{\parskip}{6pt plus 2pt minus 1pt}}
}{% if KOMA class
  \KOMAoptions{parskip=half}}
\makeatother
\usepackage{xcolor}
\usepackage[margin=1in]{geometry}
\usepackage{color}
\usepackage{fancyvrb}
\newcommand{\VerbBar}{|}
\newcommand{\VERB}{\Verb[commandchars=\\\{\}]}
\DefineVerbatimEnvironment{Highlighting}{Verbatim}{commandchars=\\\{\}}
% Add ',fontsize=\small' for more characters per line
\usepackage{framed}
\definecolor{shadecolor}{RGB}{248,248,248}
\newenvironment{Shaded}{\begin{snugshade}}{\end{snugshade}}
\newcommand{\AlertTok}[1]{\textcolor[rgb]{0.94,0.16,0.16}{#1}}
\newcommand{\AnnotationTok}[1]{\textcolor[rgb]{0.56,0.35,0.01}{\textbf{\textit{#1}}}}
\newcommand{\AttributeTok}[1]{\textcolor[rgb]{0.13,0.29,0.53}{#1}}
\newcommand{\BaseNTok}[1]{\textcolor[rgb]{0.00,0.00,0.81}{#1}}
\newcommand{\BuiltInTok}[1]{#1}
\newcommand{\CharTok}[1]{\textcolor[rgb]{0.31,0.60,0.02}{#1}}
\newcommand{\CommentTok}[1]{\textcolor[rgb]{0.56,0.35,0.01}{\textit{#1}}}
\newcommand{\CommentVarTok}[1]{\textcolor[rgb]{0.56,0.35,0.01}{\textbf{\textit{#1}}}}
\newcommand{\ConstantTok}[1]{\textcolor[rgb]{0.56,0.35,0.01}{#1}}
\newcommand{\ControlFlowTok}[1]{\textcolor[rgb]{0.13,0.29,0.53}{\textbf{#1}}}
\newcommand{\DataTypeTok}[1]{\textcolor[rgb]{0.13,0.29,0.53}{#1}}
\newcommand{\DecValTok}[1]{\textcolor[rgb]{0.00,0.00,0.81}{#1}}
\newcommand{\DocumentationTok}[1]{\textcolor[rgb]{0.56,0.35,0.01}{\textbf{\textit{#1}}}}
\newcommand{\ErrorTok}[1]{\textcolor[rgb]{0.64,0.00,0.00}{\textbf{#1}}}
\newcommand{\ExtensionTok}[1]{#1}
\newcommand{\FloatTok}[1]{\textcolor[rgb]{0.00,0.00,0.81}{#1}}
\newcommand{\FunctionTok}[1]{\textcolor[rgb]{0.13,0.29,0.53}{\textbf{#1}}}
\newcommand{\ImportTok}[1]{#1}
\newcommand{\InformationTok}[1]{\textcolor[rgb]{0.56,0.35,0.01}{\textbf{\textit{#1}}}}
\newcommand{\KeywordTok}[1]{\textcolor[rgb]{0.13,0.29,0.53}{\textbf{#1}}}
\newcommand{\NormalTok}[1]{#1}
\newcommand{\OperatorTok}[1]{\textcolor[rgb]{0.81,0.36,0.00}{\textbf{#1}}}
\newcommand{\OtherTok}[1]{\textcolor[rgb]{0.56,0.35,0.01}{#1}}
\newcommand{\PreprocessorTok}[1]{\textcolor[rgb]{0.56,0.35,0.01}{\textit{#1}}}
\newcommand{\RegionMarkerTok}[1]{#1}
\newcommand{\SpecialCharTok}[1]{\textcolor[rgb]{0.81,0.36,0.00}{\textbf{#1}}}
\newcommand{\SpecialStringTok}[1]{\textcolor[rgb]{0.31,0.60,0.02}{#1}}
\newcommand{\StringTok}[1]{\textcolor[rgb]{0.31,0.60,0.02}{#1}}
\newcommand{\VariableTok}[1]{\textcolor[rgb]{0.00,0.00,0.00}{#1}}
\newcommand{\VerbatimStringTok}[1]{\textcolor[rgb]{0.31,0.60,0.02}{#1}}
\newcommand{\WarningTok}[1]{\textcolor[rgb]{0.56,0.35,0.01}{\textbf{\textit{#1}}}}
\usepackage{graphicx}
\makeatletter
\def\maxwidth{\ifdim\Gin@nat@width>\linewidth\linewidth\else\Gin@nat@width\fi}
\def\maxheight{\ifdim\Gin@nat@height>\textheight\textheight\else\Gin@nat@height\fi}
\makeatother
% Scale images if necessary, so that they will not overflow the page
% margins by default, and it is still possible to overwrite the defaults
% using explicit options in \includegraphics[width, height, ...]{}
\setkeys{Gin}{width=\maxwidth,height=\maxheight,keepaspectratio}
% Set default figure placement to htbp
\makeatletter
\def\fps@figure{htbp}
\makeatother
\setlength{\emergencystretch}{3em} % prevent overfull lines
\providecommand{\tightlist}{%
  \setlength{\itemsep}{0pt}\setlength{\parskip}{0pt}}
\setcounter{secnumdepth}{-\maxdimen} % remove section numbering
\ifLuaTeX
  \usepackage{selnolig}  % disable illegal ligatures
\fi
\IfFileExists{bookmark.sty}{\usepackage{bookmark}}{\usepackage{hyperref}}
\IfFileExists{xurl.sty}{\usepackage{xurl}}{} % add URL line breaks if available
\urlstyle{same}
\hypersetup{
  pdftitle={全流程整合},
  pdfauthor={Liangxiao LI},
  hidelinks,
  pdfcreator={LaTeX via pandoc}}

\title{全流程整合}
\author{Liangxiao LI}
\date{2023-07-17}

\begin{document}
\maketitle

\hypertarget{r-markdown}{%
\subsection{R Markdown}\label{r-markdown}}

First load necessary library.

\begin{Shaded}
\begin{Highlighting}[]
\FunctionTok{library}\NormalTok{(sp)}
\end{Highlighting}
\end{Shaded}

\begin{verbatim}
## The legacy packages maptools, rgdal, and rgeos, underpinning the sp package,
## which was just loaded, will retire in October 2023.
## Please refer to R-spatial evolution reports for details, especially
## https://r-spatial.org/r/2023/05/15/evolution4.html.
## It may be desirable to make the sf package available;
## package maintainers should consider adding sf to Suggests:.
## The sp package is now running under evolution status 2
##      (status 2 uses the sf package in place of rgdal)
\end{verbatim}

\begin{Shaded}
\begin{Highlighting}[]
\FunctionTok{library}\NormalTok{(spgwr)}
\end{Highlighting}
\end{Shaded}

\begin{verbatim}
## Loading required package: spData
\end{verbatim}

\begin{verbatim}
## To access larger datasets in this package, install the spDataLarge
## package with: `install.packages('spDataLarge',
## repos='https://nowosad.github.io/drat/', type='source')`
\end{verbatim}

\begin{verbatim}
## NOTE: This package does not constitute approval of GWR
## as a method of spatial analysis; see example(gwr)
\end{verbatim}

\begin{Shaded}
\begin{Highlighting}[]
\CommentTok{\#library(spdep)}
\FunctionTok{library}\NormalTok{(readxl)}
\CommentTok{\#library(spatialreg)}
\FunctionTok{library}\NormalTok{(gstat)}
\FunctionTok{library}\NormalTok{(stats)}
\FunctionTok{library}\NormalTok{(dbscan)}
\end{Highlighting}
\end{Shaded}

\begin{verbatim}
## 
## Attaching package: 'dbscan'
\end{verbatim}

\begin{verbatim}
## The following object is masked from 'package:stats':
## 
##     as.dendrogram
\end{verbatim}

\begin{Shaded}
\begin{Highlighting}[]
\FunctionTok{library}\NormalTok{(cluster)}
\FunctionTok{library}\NormalTok{(ggplot2)}
\FunctionTok{library}\NormalTok{(scatterplot3d)}
\FunctionTok{library}\NormalTok{(rgl)}
\FunctionTok{library}\NormalTok{(mclust)}
\end{Highlighting}
\end{Shaded}

\begin{verbatim}
## Package 'mclust' version 6.0.0
## Type 'citation("mclust")' for citing this R package in publications.
\end{verbatim}

\begin{Shaded}
\begin{Highlighting}[]
\FunctionTok{library}\NormalTok{(plotly)}
\end{Highlighting}
\end{Shaded}

\begin{verbatim}
## 
## Attaching package: 'plotly'
\end{verbatim}

\begin{verbatim}
## The following object is masked from 'package:ggplot2':
## 
##     last_plot
\end{verbatim}

\begin{verbatim}
## The following object is masked from 'package:stats':
## 
##     filter
\end{verbatim}

\begin{verbatim}
## The following object is masked from 'package:graphics':
## 
##     layout
\end{verbatim}

\begin{Shaded}
\begin{Highlighting}[]
\FunctionTok{library}\NormalTok{(dplyr)}
\end{Highlighting}
\end{Shaded}

\begin{verbatim}
## 
## Attaching package: 'dplyr'
\end{verbatim}

\begin{verbatim}
## The following objects are masked from 'package:stats':
## 
##     filter, lag
\end{verbatim}

\begin{verbatim}
## The following objects are masked from 'package:base':
## 
##     intersect, setdiff, setequal, union
\end{verbatim}

\begin{Shaded}
\begin{Highlighting}[]
\FunctionTok{library}\NormalTok{(FNN)}
\FunctionTok{library}\NormalTok{(matrixStats)}
\end{Highlighting}
\end{Shaded}

\begin{verbatim}
## 
## Attaching package: 'matrixStats'
\end{verbatim}

\begin{verbatim}
## The following object is masked from 'package:dplyr':
## 
##     count
\end{verbatim}

\begin{Shaded}
\begin{Highlighting}[]
\FunctionTok{library}\NormalTok{(tidyr)}
\FunctionTok{library}\NormalTok{(scales)}
\FunctionTok{library}\NormalTok{(factoextra)}
\end{Highlighting}
\end{Shaded}

\begin{verbatim}
## Welcome! Want to learn more? See two factoextra-related books at https://goo.gl/ve3WBa
\end{verbatim}

\begin{Shaded}
\begin{Highlighting}[]
\FunctionTok{library}\NormalTok{(spatstat)}
\end{Highlighting}
\end{Shaded}

\begin{verbatim}
## Loading required package: spatstat.data
\end{verbatim}

\begin{verbatim}
## Loading required package: spatstat.geom
\end{verbatim}

\begin{verbatim}
## spatstat.geom 3.2-2
\end{verbatim}

\begin{verbatim}
## 
## Attaching package: 'spatstat.geom'
\end{verbatim}

\begin{verbatim}
## The following object is masked from 'package:scales':
## 
##     rescale
\end{verbatim}

\begin{verbatim}
## The following object is masked from 'package:cluster':
## 
##     volume
\end{verbatim}

\begin{verbatim}
## Loading required package: spatstat.random
\end{verbatim}

\begin{verbatim}
## spatstat.random 3.1-5
\end{verbatim}

\begin{verbatim}
## Loading required package: spatstat.explore
\end{verbatim}

\begin{verbatim}
## Loading required package: nlme
\end{verbatim}

\begin{verbatim}
## 
## Attaching package: 'nlme'
\end{verbatim}

\begin{verbatim}
## The following object is masked from 'package:dplyr':
## 
##     collapse
\end{verbatim}

\begin{verbatim}
## spatstat.explore 3.2-1
\end{verbatim}

\begin{verbatim}
## 
## Attaching package: 'spatstat.explore'
\end{verbatim}

\begin{verbatim}
## The following object is masked from 'package:gstat':
## 
##     idw
\end{verbatim}

\begin{verbatim}
## Loading required package: spatstat.model
\end{verbatim}

\begin{verbatim}
## Loading required package: rpart
\end{verbatim}

\begin{verbatim}
## spatstat.model 3.2-4
\end{verbatim}

\begin{verbatim}
## Loading required package: spatstat.linnet
\end{verbatim}

\begin{verbatim}
## spatstat.linnet 3.1-1
\end{verbatim}

\begin{verbatim}
## 
## spatstat 3.0-6 
## For an introduction to spatstat, type 'beginner'
\end{verbatim}

\begin{Shaded}
\begin{Highlighting}[]
\FunctionTok{library}\NormalTok{(clValid)}
\FunctionTok{library}\NormalTok{(factoextra)}
\FunctionTok{library}\NormalTok{(fpc)}
\end{Highlighting}
\end{Shaded}

\begin{verbatim}
## 
## Attaching package: 'fpc'
\end{verbatim}

\begin{verbatim}
## The following object is masked from 'package:dbscan':
## 
##     dbscan
\end{verbatim}

\begin{Shaded}
\begin{Highlighting}[]
\FunctionTok{library}\NormalTok{(NbClust)}
\FunctionTok{library}\NormalTok{(readxl)}
\FunctionTok{library}\NormalTok{(plot3D)}
\end{Highlighting}
\end{Shaded}

\hypertarget{including-plots}{%
\subsection{Including Plots}\label{including-plots}}

You can also embed plots, for example:

\includegraphics{全流程整合_files/figure-latex/pressure-1.pdf}

Note that the \texttt{echo\ =\ FALSE} parameter was added to the code
chunk to prevent printing of the R code that generated the plot.

\end{document}
